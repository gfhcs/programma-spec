\chapter{The language} \label{chp:language}

Having put into place all the necessary definitions, we can now specify the actual Programma language:

\section{Type names} \label{sec:type_names}

\begin{bnf}

  <\E{type}> ::= \todo {Specify this!}

\end{bnf}

\todo{Add static rules to assign these phrases the type they refer to!}

\section{Expressions} \label{sec:expressions}

The following grammar specifies how expressions are to be formed:

\begin{bnf}

  <exp>   ::= <eq-check> ["?" <exp> ":" <exp>]
  
  <eq-check>  ::= <ty-check> [(\R{$(\langle eq-check \rangle , \lit{=}, \langle eq-pos \rangle) $}{"="} |\R{$(\langle eq-check \rangle , \lit{≠}, \langle eq-neg \rangle) $}{"≠"}) <exp>]
  
  <ty-check>  ::= <call> [(\R{$(\langle ty-check \rangle , \lit{$\in$}, \langle ty-pos \rangle) $}{\lit*{$\in$}}|\R{$(\langle ty-check \rangle , \lit*{$\notin$}, \langle ty-neg \rangle) $}{\lit*{$\notin$}}) <type>]
  
  <call>  ::= <projection> <call'>  | \R{$(\langle call \rangle , \lit{new}, \langle \E{creation} \rangle) $}{"new"} <id> <call'>
  
  <call'> ::= [<projection> <call'>]
  
  <projection>  ::= <primitive> <projection'>

  <projection'> ::= [\lit{[} <exp> [":" <exp>] \lit{]} <projection'>]

  <primitive> ::= <bits> | <symbols> | <id> | \R{$(\langle primitive \rangle , \lit{λ}, \langle \E{abs} \rangle) $}{"λ"} <parameters> <block> | \lit{(} <exp> \lit{)}
  
  <parameter> ::= <type> <id>

  <parameters> ::= [<parameter> <parameters>]

\end{bnf}

Given a \define{type context } $T : \setS \rightarrow Types$, the \define{static semantics} of these expressions are defined by the following inference rules:

\begin{column}{0.24}

	\irule {\lb i {bits}}{\eval T i bit}

\nextColumn{0.25}
	
	\irule {\lb i {symbols}}{\eval T i symbol}

\nextColumn {0.49}

	\irule {\lb i {id} \ws i \rightarrow j \ws \eval T {l(j)} t}{\eval T i t}

\end{column}

\begin{column}{0.3}
\irule{\lb i {parameter} \\ \eval T {i[0]} t}{\eval T i t}

\nextColumn {0.37}

\irule{\lb i {parameters} \\ \eval T {i[0]} t \ws  \eval T {i[1]} t'}{\eval T i t \rightarrow t'}
\nextColumn {0.3}

\end{column}

For $i \in \setS$ with $l(i) \in \{\langle parameter\rangle, \langle parameters\rangle\}$ and a type context $T$ we define $e(T) : \setS \rightarrow (\setS \rightarrow \wp (Types))$ by

\begin{column}{0.3}

\irule{\lb i {parameter} \\ \eval T {i[0]} t \ws lw(i[1]) = p}{e(T)(i) = T[p := t]}

\todo{Guess we should make sure that there aren't two equally named parameters! (Cause that's nonsense.)}

\irule{\lb i {parameters} \\ \eval e(T)(i[0]) = T' \ws e(T')(i[1]) = T''}{e(T)(i) = T''}

\end{column}


\begin{column}{0.3}

\irule{\lb i {abs} \ws }{\eval T i {t \rightarrow t'}}

\todo{We need to assign parameters their types!}
\irule{\lb i {abs} \ws \eval T {i[0]} t \ws \eval {T[]} {i[1]} t'} {\eval T i {t \rightarrow t'}}

\end{column}


%Calls work like that: Let $x$ be an instance of type $t$, $m$ a method of $t$ and $a : u$ and argument for $m$. Then the call
%
%\[ x \; m \; a\]
%
%could be typed as 
%\[x : (t \rightarrow \alpha) \rightarrow \alpha\]
%\[m : (t \rightarrow u \rightarrow u\]


\section{Statements} \label{sec:statements}




\[t (S, \rightarrow_2 , s_0) := (S, \{ s  \stackrel {A} {\rightarrow_1}  s'  \mid A \neq \emptyset \wedge  (\forall \alpha \in A : s \stackrel {\alpha} {\rightarrow_2}  s') \}, s_0)\]



\begin{bnf}

  <\E{block}> ::= \todo {Specify this!}

\end{bnf}

