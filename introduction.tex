\chapter{Introduction}

The reader will certainly ask why it should be necessary to introduce another programming
language. This question shall be answered here.

As computer science and computers themselves improved over the decades the
programming languages evolved to very sophisticated tools allowing us to very comfortably
tell a computer what is to be done.

One major feature that most programming languages have is their universality: The fact
that they are Turing complete ensures that we can describe any computational task using
these languages.

The problem Programma aims to solve is the lack of universality of a very different kind:
Programmers put a great deal of effort into writing code that is robust, fast and
serviceable. Having accomplished this task however, there remains an unpleasant
uncertainty about the program code: For how long will it keep its value. How long will it
take until no one is able to execute this code either because of lack of knowledge or
because of lack of a suitable runtime environment?

To sum these questions up one can say that the ``time universality'' of code written in these
languages is not given. The code is not ``eternal''. It works here and now. But it probably
won't work tomorrow at another place because technology changes.

The goal of the Programma project is to provide this ``eternal universality''. It tries to
achieve it by reducing the assumptions made concerning the runtime environment: The
``computer'' is abstracted away and replaced by a more general ``executive instance''.
Programma does not rely on a 32 bit or a 64 bit architecture nor does it require the
availability of a network connection or a hard drive and the like. Programma is designed to avoid
unnecessary requirements or determinations about the executive instance and its abilities
which causes known concepts like bytes, files, or a console to be abandoned.

It remains to be seen if this ambitious approach to programming will be performant or
even feasible and if it is suited to provide ``eternal universality'' for the programs we write.